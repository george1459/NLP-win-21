\documentclass{article}
\usepackage[utf8]{inputenc}
\usepackage[english]{babel}
\usepackage[]{amsthm} %lets us use \begin{proof}
\usepackage[]{amssymb} %gives us the character \varnothing
\usepackage{amsmath}
%\usepackage[shortlabels]{enumitem}
%\usepackage{CJKutf8}
\usepackage{float}
\usepackage{booktabs}

\title{NLP Final Project: A Literature Review}
\author{Shicheng Liu, Qitian Hu, Yijie Yao}

\linespread{1.5}
\usepackage[margin=1.5in]{geometry}
\usepackage{graphicx}

\begin{document}
\maketitle 

\section{Minimal Requirement}


\section{Related Ideas and Methodological Extension}


There are a lot of ways by which we could build on the existing paper, and in this section I outline several directions we would like to explore.

\subsection{Other ways of finding ideas}

While the original paper provides us an interesting framework of analyzing the relation between ideas, it has used the standard topic modeling (LDA) to identify ideas. We think there are several weak points that it could be modified:

\begin{enumerate}
  \item There are randomness involved in this process and the topics we identify rely on the training process and random seed.
  \item We cannot incorporate prior knowledge on the corpus and the ideas in it. Since topic modeling is an unsupervised method, we cannot be sure that we will guaranteed to have some topics that we're interested in. 
\end{enumerate}





\section{Possible Research Questions}







% \begin{thebibliography}{999}

% \bibitem{1}
%   cite if needed

% \end{thebibliography}








\end{document}